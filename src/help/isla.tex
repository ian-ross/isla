\documentclass[11pt,makeidx,texhelp]{report}

\usepackage[a4paper,margin=1in]{geometry}
\usepackage{fourier}

\parskip=10pt
\parindent=0pt

\title{Isla v1.0: the UM island file editor}
\author{Ian Ross}
\date{13 May 2013}

\makeindex

\begin{document}

\maketitle

\tableofcontents

%======================================================================
\chapter*{Copyright notice}

Copyright (c) 2013 Ian Ross.

Permission to use, copy, modify, and distribute this software and its
documentation for any purpose is hereby granted without fee, provided that the
above copyright notice, author statement and this permission notice appear in
all copies of this software and related documentation.

THE SOFTWARE IS PROVIDED ``AS-IS'' AND WITHOUT WARRANTY OF ANY KIND, EXPRESS,
IMPLIED OR OTHERWISE, INCLUDING WITHOUT LIMITATION, ANY WARRANTY OF
MERCHANTABILITY OR FITNESS FOR A PARTICULAR PURPOSE.

IN NO EVENT SHALL IAN ROSS BE LIABLE FOR ANY SPECIAL, INCIDENTAL,
INDIRECT OR CONSEQUENTIAL DAMAGES OF ANY KIND, OR ANY DAMAGES
WHATSOEVER RESULTING FROM LOSS OF USE, DATA OR PROFITS, WHETHER OR NOT
ADVISED OF THE POSSIBILITY OF DAMAGE, AND ON ANY THEORY OF LIABILITY,
ARISING OUT OF OR IN CONNECTION WITH THE USE OR PERFORMANCE OF THIS
SOFTWARE.

%======================================================================
\chapter{Introduction}

Isla is a wxWidgets GUI application for generating Met Office UM
island data (used for streamfunction line integral calculations in the
barotropic solver of the ocean model) from land/sea masks.  The basic
workflow for using Isla is as follows:
\begin{enumerate}
  \item{The user generates a NetCDF file containing the land/sea mask
    of interest.}
  \item{The user starts Isla and loads the NetCDF land/sea mask.}
  \item{From the land/sea mask, Isla:
    \begin{itemize}
      \item{determines land masses;}
      \item{classifies the land masses as island or not island
        depending on area;}
      \item{calculates valid segmentations for each island following
        the requirements of the Met Office UM ocean model.}
    \end{itemize}}
  \item{The user can then:
    \begin{itemize}
      \item{review the island segmentations;}
      \item{optionally specify that selected land masses are or are
        not islands, independent of the area threshold used by Isla;}
      \item{refine or coarsen the island segmentations calculated by
        Isla for efficiency.}
    \end{itemize}}
  \item{The user can also load island data from a UM ocean dump file
    or from an ASCII file (in the format used by \texttt{mkancil}) for
    comparison with the island segmentation calculated by Isla.}
  \item{Once satisfied with the island segmentation, the user can
    export the segmentation to an ASCII file for processing with
    \texttt{mkancil}.}
\end{enumerate}


%======================================================================
\chapter{Map View and Navigation}

\section{Map view}

The main window view in Isla displays a map showing the land/sea mask,
island segment outlines (both those calculated from the land/sea mask
and optionally island segments loaded from a dump file or ASCII island
file for comparison).  Map boundaries show latitude and longitude
values (bottom and right) and model tracer grid cell coordinate values
(top and left).

At large scales, no grid lines are displayed to avoid clutter; zooming
in far enough causes grid lines to be displayed between grid cells.

Once a land/sea mask is loaded, landmasses are displayed using the
colours defined in the preferences dialogue.  By default, ocean grid
cells are white, island land cells are black and non-island land cells
are grey.

If the ``Show islands'' option in the \textbf{View} menu is selected
(on by default), then island segments calculated by Isla are displayed
(by default, as orange lines).  Adjacent segments from the same island
are ``stitched together'' with hatching to distinguish them from
segments belonging to different islands.

If the ``Show comparison'' option in the \textbf{View} menu is
selected (on by default), then if island comparison data is loaded,
the comparison island segments are displayed (by default, as dashed
blue lines).  As for the calculated island segments, adjacent segments
from the same island are ``stitched together'' with hatching to
distinguish them from segments belonging to different islands.


\section{Navigation}

The map view can be scrolled by a number of mechanisms:
\begin{itemize}
  \item{Dragging or mousewheeling on the horizontal axes pans the map
    in a zonal direction, dragging or mousewheeling on the vertical
    axes pans the map in a meridional direction;}
  \item{Mousewheeling in the main area of the map pans the map in a
    meridional direction;}
  \item{The arrow keys move the map in the relevant direction, one
    grid cell at a time;}
  \item{If the ``pan'' tool is selected (the four-pointed arrow icon,
    also available as ``Pan'' from the \textbf{View} menu or by
    pressing the \textit{P} key), then dragging anywhere in the map
    window pans the view.}
\end{itemize}

In all cases, scrolling in longitude is periodic.

The map view resizes to fit the application window, and the view scale
can be controlled using the zoom tools:
\begin{itemize}
  \item{Zoom in (toolbar: magnifying glass with plus sign, ``Zoom in''
    in the \textbf{View} menu, or the \textit{+} key);}
  \item{Zoom out (toolbar: magnifying glass with minus sign, ``Zoom
    out'' in the \textbf{View} menu, or the \textit{-} key);}
  \item{Zoom to fit, which rescales the map to fit the entire globe
    into the window (toolbar: magnifying glass with black frame,
    ``Zoom to fit'' in the \textbf{View} menu, or the \textit{F}
    key);}
  \item{Zoom to a selection, which allows the user to drag over a
    region of the map to be zoomed (toolbar: magnifying glass with
    blue frame, ``Zoom to selection'' in the \textbf{View} menu, the
    \textit{Z} key).}
\end{itemize}


%======================================================================
\chapter{Menu operations}

\section{File menu}

The \textbf{File} menu has the following operations:
\begin{description}
  \item[Clear grid (\textit{Ctrl-N})]{Clear all island and land/sea
    mask information.}
  \item[Load land/sea mask... (\textit{Ctrl-O})]{Loads a land/sea mask
    from a NetCDF file; triggers determination of landmasses, island
    thresholding and island segmentation calculations.}
  %% \item[Save land/sea mask... (\textit{Ctrl-S})]{Writes the current
  %%   land/sea mask to a new NetCDF file.}
  \item[Export island data... (\textit{Ctrl-E})]{Saves current island
    segments to ASCII island file (suitable for use with recent
    versions of \texttt{mkancil}).}
  \item[Import island comparison data... (\textit{Ctrl-I})]{Loads
    island data either from a UM dump file or from an ASCII island
    file -- the island segments for the comparison data are displayed
    in a style that distinguishes them from the island segments
    calculated by Isla.}
  \item[Clear island comparison data (\textit{Ctrl-C})]{Clears any
    island comparison data that has been loaded.}
  \item[Preferences]{Open preferences dialogue, where colours and
    other options can be set (the only really relevant one at the
    moment is the land surface area used as the threshold for deciding
    whether a landmass defaults to being an island).}
  \item[Quit (\textit{Ctrl-Q})]{Close the Isla program.}
\end{description}

\section{View menu}

\begin{description}
  \item[Zoom in (\textit{+})]{Zoom in by a fixed factor (up to a
    maximum magnification level).}
  \item[Zoom out (\textit{-})]{Zoom out by a fixed factor (to a fixed
    minimum scale or until the whole map is visible).}
  \item[Zoom to fit (\textit{F})]{Zoom to fit the whole map into the
    window.}
  \item[Zoom to selection (\textit{Z})]{Zoom to a region selected by
    dragging a ``rubberband'' box on the map.}
  \item[Pan (\textit{P})]{Switch pan tool on/off -- when the pan tool
    is on, dragging anywhere in the map view pans the map.}
  \item[Show islands (\textit{I})]{Toggle display of island
    segmentation information calculated by Isla.}
  \item[Show comparison (\textit{C})]{Toggle display of comparison
    island segmentation information loaded from a dump or ASCII file.}
\end{description}

%% \section{Tools menu}

%% The two items in this menu are really only of interest for testing
%% Isla: by default, the ``Select'' tool is engaged.  The other option,
%% ``Edit land/sea mask'', is used for modifying land/sea masks to be
%% saved to new NetCDF files for testing purposes.  Most normal uses of
%% Isla should not need this tool.

\section{Help menu}

Documentation for Isla is accessed from the ``Contents'' item in the
\textbf{Help} menu.  The ``About'' option displays version
information.

%% \section{Debug menu}

%% Isla v0.2 also has a \textbf{Debug} menu that allows information used
%% in the island segmentation computation to be displayed as overlays on
%% the map view.  Of most interest is the ``ISMASK overlay'' option,
%% which displays (on the model streamfunction grid) the \texttt{ISMASK}
%% field used in the UM streamfunction line integral calculation
%% (\texttt{ISMASK=0} corresponds to ocean points, \texttt{ISMASK=1} to
%% landmass boundary points and \texttt{ISMASK=2} to landmass interior
%% points).  Other options display land mass assignments and island
%% determination.  These tools are probably only of interest to users
%% helping to debug problems with Isla.


%======================================================================
\chapter{Isla Calculations}

\section{File types}

Isla uses three main file types: NetCDF land/sea masks, UM ocean model
dump files and ASCII island files.

\subsection{NetCDF land/sea masks}

The initial step in any use of Isla is to load a land/sea mask from a
NetCDF file.  These files should be defined to have two dimensions
called either ``lat'' and ``lon'' or ``latitude'' and ``longitude''.
The latitude and longitude dimensions should have coordinate
variables, following CF metadata conventions, and the longitude
dimension should have a number of points corresponding to the true
number of longitude points in the model tracer grid.  Some care is
required here, because \texttt{xconv}, for instance, writes NetCDF
files that include the two ``ghost'' longitude points used by the UM
ocean model to deal with zonal wraparound.  These ghost points should
be cut off, using \texttt{ncks} or \texttt{cdo}, before attempting to
load the land/sea mask into Isla.

The actual mask variable in the input NetCDF file can have any name
and be of any numeric or logical type -- non-zero values correspond to
land points, zero values to ocean.

\subsection{UM ocean dump files}

Isla is able to read island data from UM ocean dump files for
comparison with calculated island segments.  Only ocean dump files in
64-bit IEEE format are supported (although endianness doesn't
matter).  The ocean dump file grid must (obviously) match the grid on
which the land/sea mask is defined.

\subsection{ASCII island files}

ASCII island files are supported both for the loading of comparison
data and for export of island segmentation data for further processing
by \texttt{mkancil}.

The ASCII island file format is very simple.  Lines beginning with a
hash (\texttt{\#}) character are comments.  All other lines should
consist of integer values:
\begin{itemize}
  \item{The first integer in the file is the number of islandsm
    \texttt{Ni}.}
  \item{For each island, there is:
    \begin{itemize}
      \item{a segment count \texttt{Ns}, then}
      \item{segment start and end grid cell indexes for each island
        segment, in the order \texttt{IS(1)..IS(Ns)},
        \texttt{IE(1)..IE(Ns)}, \texttt{JS(1)..JS(Ns)},
        \texttt{JE(1)..JE(Ns)}, where the \texttt{I} variables are
        column indexes, the \texttt{J} variables are row indexes, the
        \texttt{S} variables are segment start indexes and the
        \texttt{E} variables are segment end indexes.}
    \end{itemize}}
\end{itemize}

\section{Isla processing}

When a land/sea mask is loaded, Isla performs the following
calculations:
\begin{enumerate}
  \item{Individual landmasses are identified using a ``flood fill''
    algorithm.  The separation between landmasses is defined on the UM
    ocean model streamfunction grid, which means that diagonally
    adjacent cells are considered to belong to the same landmass.}
  \item{Landmasses whose surface area is below a configurable
    threshold are identified as islands (the default threshold is
    8,000,000 sq. km, which is a bit more than the area of
    Australia).}
  \item{For each island landmass, a set of bounding segments
    (rectangles) is calculated.  The details of the algorithm used are
    a little complicated (look in \texttt{IslaCompute.cpp} if you're
    interested), but the result is, as far as possible, a minimal set
    of admissible segments for each island, minimal in the sense of
    using the smallest possible number of segments, and admissible in
    the sense that the union of the segments for an island contains
    all points for that island with \texttt{ISMASK=1} or
    \texttt{ISMASK=2} and no points for any other landmasses that have
    \texttt{ISMASK=1} or \texttt{ISMASK=2}.}
  \item{Polar islands are optimised by moving their outer boundaries
    as far as possible towards the equator to remove lines of latitude
    that are exclusively cells with \texttt{ISMASK=2} (which are not
    required in the streamfunction line integral calculations in the
    UM ocean model).}
\end{enumerate}

\section{User modification of island segmentation}

Once the initial island segmentation calculation has been completed,
the user can modify the results by right clicking on landmasses and
selecting from the popup menu.  The options are:

\begin{description}
  \item[Landmass island toggle]{Toggle the selected land mass between
    being considered as an island or not.  All land masses treated as
    islands are segmented and their segments written into ASCII island
    export files.  This functionality may be useful in a number of
    different cases: some simulations treat the Americas as an
    ``island'', but the land mass is too large to be considered an
    island by default by Isla; some paleo applications may have large
    isolated landmasses (Pangaea) that need to be treated as islands.}
  \item[Coarsen island segmentation]{Attempt to merge segments for the
    selected island to produce a segmentation with less segments, but
    probably more redundant ocean points.}
  \item[Refine island segmentation]{Attempt to split segments for the
    selected island to produce a segmentation with more segments, but
    probably less redundant ocean points.}
  \item[Reset island segmentation]{Reruns the island segmentation for
    the selected island with the default Isla settings.}
\end{description}

The idea of the ``coarsen''/``refine'' functions is that, for an
isolated island far from any other landmasses, the simplest
segmentation is a single segment -- the bounding box of the island.
In some cases, this may end up being rather inefficient in the
streamfunction line integral calculations in the UM ocean model.  For
example, consider a long thin diagonally oriented island (like New
Zealand).  Using a single rectangular segment to cover such an island
will produce correct results in the streamfunction line integrals
(since all the island boundary points are included in the
segmentation), but will also include a large number of ocean points
that do not participate in the line integral calculations but still
have to be examined to check their \texttt{ISMASK} values.  Splitting
the segmentation for such an island into two (or more) segments may be
more efficient, simply because the segments can be shrunk to contain
less ocean points.  There is a trade-off between needing to loop over
more or less island segments and needing to consider less or more
redundant ocean points.  The ``coarsen''/``refine'' functions allow a
user to produce segmentations that lie anywhere along this axis, as
required.

\addcontentsline{toc}{chapter}{Index}

\end{document}
